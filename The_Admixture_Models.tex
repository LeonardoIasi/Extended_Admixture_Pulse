\documentclass[]{article}
\usepackage{lmodern}
\usepackage{amssymb,amsmath}
\usepackage{ifxetex,ifluatex}
\usepackage{fixltx2e} % provides \textsubscript
\ifnum 0\ifxetex 1\fi\ifluatex 1\fi=0 % if pdftex
  \usepackage[T1]{fontenc}
  \usepackage[utf8]{inputenc}
\else % if luatex or xelatex
  \ifxetex
    \usepackage{mathspec}
  \else
    \usepackage{fontspec}
  \fi
  \defaultfontfeatures{Ligatures=TeX,Scale=MatchLowercase}
\fi
% use upquote if available, for straight quotes in verbatim environments
\IfFileExists{upquote.sty}{\usepackage{upquote}}{}
% use microtype if available
\IfFileExists{microtype.sty}{%
\usepackage{microtype}
\UseMicrotypeSet[protrusion]{basicmath} % disable protrusion for tt fontshttps://de.overleaf.com/project/5e85b0680d0bed00011ea790
}{}
\usepackage[margin=1in]{geometry}
\usepackage{hyperref}
\hypersetup{unicode=true,
            pdftitle={The dating of the Human-Neandertal introgression event estimated from present-day human genomes is compatible with a multitude of admixture durations},
            pdfauthor={Leonardo Nicola Martin Iasi (Max Planck Institute for Evolutionary Anthropology, MPI EVA), Dr.~Benjamin Marco Peter (MPI EVA, benjamin\_peter@eva.mpg.de)},
            pdfborder={0 0 0},
            breaklinks=true}
\urlstyle{same}  % don't use monospace font for urls
\usepackage{natbib}
\bibliographystyle{plainnat}
\usepackage{graphicx,grffile}
\makeatletter
\def\maxwidth{\ifdim\Gin@nat@width>\linewidth\linewidth\else\Gin@nat@width\fi}
\def\maxheight{\ifdim\Gin@nat@height>\textheight\textheight\else\Gin@nat@height\fi}
\makeatother
% Scale images if necessary, so that they will not overflow the page
% margins by default, and it is still possible to overwrite the defaults
% using explicit options in \includegraphics[width, height, ...]{}
\setkeys{Gin}{width=\maxwidth,height=\maxheight,keepaspectratio}
\IfFileExists{parskip.sty}{%
\usepackage{parskip}
}{% else
\setlength{\parindent}{0pt}
\setlength{\parskip}{6pt plus 2pt minus 1pt}
}
\setlength{\emergencystretch}{3em}  % prevent overfull lines
\providecommand{\tightlist}{%
  \setlength{\itemsep}{0pt}\setlength{\parskip}{0pt}}
\setcounter{secnumdepth}{0}
% Redefines (sub)paragraphs to behave more like sections
\ifx\paragraph\undefined\else
\let\oldparagraph\paragraph
\renewcommand{\paragraph}[1]{\oldparagraph{#1}\mbox{}}
\fi
\ifx\subparagraph\undefined\else
\let\oldsubparagraph\subparagraph
\renewcommand{\subparagraph}[1]{\oldsubparagraph{#1}\mbox{}}
\fi

%%% Use protect on footnotes to avoid problems with footnotes in titles
\let\rmarkdownfootnote\footnote%
\def\footnote{\protect\rmarkdownfootnote}

%%% Change title format to be more compact
\usepackage{titling}

% Create subtitle command for use in maketitle
\providecommand{\subtitle}[1]{
  \posttitle{
    \begin{center}\large#1\end{center}
    }
}

\setlength{\droptitle}{-2em}

  \title{The dating of the Human-Neandertal introgression event estimated from present-day human genomes is compatible with a multitude of admixture durations}
    \pretitle{\vspace{\droptitle}\centering\huge}
  \posttitle{\par}
    \author{Leonardo Nicola Martin Iasi (Max Planck Institute for Evolutionary
Anthropology, MPI EVA), Dr.~Benjamin Marco Peter (MPI EVA,
\href{mailto:benjamin_peter@eva.mpg.de}{\nolinkurl{benjamin\_peter@eva.mpg.de}})}
    \preauthor{\centering\large\emph}
  \postauthor{\par}
      \predate{\centering\large\emph}
  \postdate{\par}
    \date{2020-03-24}

\usepackage{setspace}
\doublespacing
\usepackage[none]{hyphenat}
\usepackage{amsfonts}
\usepackage{amssymb}
\usepackage{graphicx}
\usepackage{float}
\usepackage{xcolor}
\floatplacement{figure}{H}

\begin{document}
\maketitle

\subsubsection{Admixture Models}\label{admixture models}

In the admixture pulse model, gene flow is a one generation long pulse,
resulting in an exponentially distributed segment length or admixture-induced linkage
disequilibrium (ALD) decay curve (Eq. \ref{eq:1}), with $\lambda$ as
the rate parameter of the Exponential distribution holding the  time since the admixture event $t$ with $l$, here being either the genetic length of an admixture segment or the genetic
distance \(d\) between two SNPs,

\begin{equation}
\begin{split}
\label{eq:1}
L_i &\sim exp(\lambda) \\
\mathbb{E}[l] &= \frac{1}{\lambda} = \frac{1}{t}
\end{split}
\end{equation}

Continuous gene flow $m$ over time with the random variable $t \in \{t_1,t_2,...,t_n\}$ was modeled as a Gamma distribution (Eq.
\ref{eq:2})

\begin{equation}
\label{eq:2}
m_i \sim \Gamma(k,\theta)
\end{equation}

Where k is the shape and \(\theta\) the scale parameter. The parameter
values are chosen such that the mean length of the
exponentially distributed segment length or ALD decay curve resulting from the one
generation admixture pulse, is equal to the mean length as a result of continuous migration with the same total amount of
migrants, modeled using a Lomax distribution with shape being $k+1$ and the scale $\frac{1}{\theta}$ (Eq. \ref{eq:3})

\begin{equation}
\begin{split}
\label{eq:3}
L_i &\sim Lomax(k+1,\frac{1}{\theta}) \\
\mathbb{E}[l] &= \frac{1}{\lambda} = \frac{1}{k\theta}
\end{split}
\end{equation}

\begin{equation}
\label{eq:4}
t_{m}=\mathbb{E}[m] = k \theta
\end{equation}

\begin{equation*}
\nonumber
where \qquad k=t_m \, \frac{1}{\theta} \qquad and \qquad \theta=\frac{1}{t_{m}Var[m]}
\end{equation*}

Equation (Eq. \ref{eq:4}) shows the relationships between the
distribution parameters such that the resulting decay mean length are
equal. Here  $t_{m}$ is the mean time of admixture equal to the mean of the Gamma distribution ($\mathbb{E}[m]$) and $Var[m]=(\frac{t_d}{4})^2$ its variance with $t_d$ as the duration of admixture in generations.

\subsection{Theoretical framework for continuous admixture}\label{theoretical framework for continuous admixture}

First we want to establish the model of continuous admixture and an
expectation of the segment length distribution under this model in an
ideal-case from perfect data. For this purpose, we assume that the
lengths of introgressed segments are perfectly known. In this case, under
some models, the distribution of introgressed segment lengths \(L_i\) can
be written as

\begin{equation}
L_i \sim exp(t)
\end{equation}

where $t$ is the time when the segments entered the population. The mean segments length will vary  depending on the model assumptions and
recombination rate \(r\) \citep{liang_lengths_2014}. E.g under the SMC,
\(L_i = t(1-m)\), and under the SMC' allowing for back coalescence,
\(L_i = 2N(1-m)(1-exp^{-t/2N})\). For Neandertal admixture where
\(m\), the admixture fraction, is typically low, the exponential
assumption is satisfied \citep{liang_lengths_2014}. For scenarios where
the length of admixture tracts is not exponential, e.g. because
admixture is recent or very old, our results do not apply.

It is widely assumed that Neandertal ancestry entered the modern human
population over a very short period (one-generation admixture pulse). As an alternative model describing multi-generation admixture, we need to
consider \(t\) not as a single point in time, but as a random variable
itself that follows a mixture distribution \(\mathcal{D}_t\). The most
widely studied is a small number of discrete ``pulses'' of admixture, in
which case \(\mathcal{D}_t\) is categorical. Here, we instead assume a
continuous multi-generation admixture \(\mathcal{D}_t\); more precisely we assume \(\mathcal{D}_t\)
follows a \(\Gamma(k, \theta)\)-distribution. This has a number of
advantages:

\begin{itemize}
    \item We just need two parameter $k$ and $\theta$, that can be interpreted as the duration of gene flow, instead of the minimal of two additional parameters required for the pulse model.
    \item The segment length distribution $L_i$ follows a Lomax-distribution, i.e. has the analytical density $Pr(L=l) = \frac{1}{\theta k-1} (1+l\frac{1}{\theta})^{-(k-1)}$
    \item The mean segment-length is  $\frac{1}{k \theta}$ for all $k > 0$, and undefined otherwise. 
    \item As $k$ approaches infinity, we recover the exponential distribution. Thus, if segment length are directly inferred, one can use a likelihood-ratio test to distinguish continuous from discrete gene flow. As the special case of exponentially lies on the boundary of the parameter space, the test-statistic does not follow a $\chi^2$-distribution \citep{Kozubowski_Testing_2008}. Model selection using a LRT is however not possible when segment length are indirectly inferred by ALD.
\end{itemize}

Using this model we i) examining the effect of continuous admixture on
the admixture time estimates calculated using the Exponential model
assuming a pulse like admixture. ii) comparing this effect to the
effects by demography, recombination rate and analysis parameters used
for the indirect inference of admixture segment length using ALD, namely
the ascertainment scheme and the minimal distance between SNPs. iii) are
interested under which conditions the parameters of the
Lomax-distribution can be estimated accurately for a scenario of
Neandertal admixture and if it is possible to distinguish a pulse-like
admixture event (resulting in exponentially-distributed ALD)
from a continuous event (resulting in Lomax-distributed ALD).


\end{document}