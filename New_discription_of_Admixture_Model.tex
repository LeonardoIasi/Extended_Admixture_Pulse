\documentclass[]{article}
\usepackage{lmodern}
\usepackage{amssymb,amsmath}
\usepackage{ifxetex,ifluatex}
\usepackage{fixltx2e} % provides \textsubscript
\ifnum 0\ifxetex 1\fi\ifluatex 1\fi=0 % if pdftex
  \usepackage[T1]{fontenc}
  \usepackage[utf8]{inputenc}
\else % if luatex or xelatex
  \ifxetex
    \usepackage{mathspec}
  \else
    \usepackage{fontspec}
  \fi
  \defaultfontfeatures{Ligatures=TeX,Scale=MatchLowercase}
\fi
% use upquote if available, for straight quotes in verbatim environments
\IfFileExists{upquote.sty}{\usepackage{upquote}}{}
% use microtype if available
\IfFileExists{microtype.sty}{%
\usepackage{microtype}
\UseMicrotypeSet[protrusion]{basicmath} % disable protrusion for tt fontshttps://de.overleaf.com/project/5e85b0680d0bed00011ea790
}{}
\usepackage[margin=1in]{geometry}
\usepackage{hyperref}
\hypersetup{unicode=true,
            pdftitle={The dating of the Human-Neandertal introgression event estimated from present-day human genomes is compatible with a multitude of admixture durations},
            pdfauthor={Leonardo Nicola Martin Iasi (Max Planck Institute for Evolutionary Anthropology, MPI EVA), Dr.~Benjamin Marco Peter (MPI EVA, benjamin\_peter@eva.mpg.de)},
            pdfborder={0 0 0},
            breaklinks=true}
\urlstyle{same}  % don't use monospace font for urls
\usepackage{natbib}
\bibliographystyle{plainnat}
\usepackage{graphicx,grffile}
\makeatletter
\def\maxwidth{\ifdim\Gin@nat@width>\linewidth\linewidth\else\Gin@nat@width\fi}
\def\maxheight{\ifdim\Gin@nat@height>\textheight\textheight\else\Gin@nat@height\fi}
\makeatother
% Scale images if necessary, so that they will not overflow the page
% margins by default, and it is still possible to overwrite the defaults
% using explicit options in \includegraphics[width, height, ...]{}
\setkeys{Gin}{width=\maxwidth,height=\maxheight,keepaspectratio}
\IfFileExists{parskip.sty}{%
\usepackage{parskip}
}{% else
\setlength{\parindent}{0pt}
\setlength{\parskip}{6pt plus 2pt minus 1pt}
}
\setlength{\emergencystretch}{3em}  % prevent overfull lines
\providecommand{\tightlist}{%
  \setlength{\itemsep}{0pt}\setlength{\parskip}{0pt}}
\setcounter{secnumdepth}{0}
% Redefines (sub)paragraphs to behave more like sections
\ifx\paragraph\undefined\else
\let\oldparagraph\paragraph
\renewcommand{\paragraph}[1]{\oldparagraph{#1}\mbox{}}
\fi
\ifx\subparagraph\undefined\else
\let\oldsubparagraph\subparagraph
\renewcommand{\subparagraph}[1]{\oldsubparagraph{#1}\mbox{}}
\fi

%%% Use protect on footnotes to avoid problems with footnotes in titles
\let\rmarkdownfootnote\footnote%
\def\footnote{\protect\rmarkdownfootnote}

%%% Change title format to be more compact
\usepackage{titling}

% Create subtitle command for use in maketitle
\providecommand{\subtitle}[1]{
  \posttitle{
    \begin{center}\large#1\end{center}
    }
}

\setlength{\droptitle}{-2em}

  \title{The dating of the Human-Neandertal introgression event estimated from present-day human genomes is compatible with a multitude of admixture durations}
    \pretitle{\vspace{\droptitle}\centering\huge}
  \posttitle{\par}
    \author{Leonardo Nicola Martin Iasi (Max Planck Institute for Evolutionary
Anthropology, MPI EVA), Dr.~Benjamin Marco Peter (MPI EVA,
\href{mailto:benjamin_peter@eva.mpg.de}{\nolinkurl{benjamin\_peter@eva.mpg.de}})}
    \preauthor{\centering\large\emph}
  \postauthor{\par}
      \predate{\centering\large\emph}
  \postdate{\par}
    \date{2020-03-24}

\usepackage{setspace}
\doublespacing
\usepackage[none]{hyphenat}
\usepackage{amsfonts}
\usepackage{amssymb}
\usepackage{graphicx}
\usepackage{float}
\usepackage{xcolor}
\floatplacement{figure}{H}

\begin{document}
\maketitle

\subsection{Admixture Models}\label{admixture models}

In this section we want to define the admixture model in terms of its segment-length distribution $l$ for a random variable $T$ giving the admixture time $t_{admix}$ over the time $t$. 


\begin{equation}
\label{eq:standard_likelihood_definintion}
    P(L=l)=\int_{0}^{\infty} P(T=t) P(L=l | T=t) \ dt
\end{equation}

\subsubsection{The Simple Pulse Model}\label{The Simple Pulse Model}

In the simple pulse model the probability of admixture over time $t$ is given by:

\begin{equation}
\label{eq:RV_simple_pulse}
  P(T)=\left\{
  \begin{array}{@{}ll@{}}
    1, & \text{if}\ T=t_{admix} \\
    0, & \text{otherwise}
  \end{array}\right.
\end{equation} 

Therefore,

\begin{equation}
\begin{split}
\label{eq:Likelihood_function_simple_pulse}
    P(L=l) &=  1  P(L=l | T=t) \\
    &= 1 te^{-tl}
\end{split}
\end{equation}

The expected segment length under a simple pulse model is given by,

\begin{equation}
\label{eq:Expected_l_simple_pulse}
    \mathbb{E}[l]&=\frac{1}{t}
\end{equation}

Note that this definition of the segment length distribution is independent from the migration rate $m$ at $t_{admix}$. Hence, the model does not consider the (higher) probability of admixture segments recombining with each other when the migration rate is high. This can be done following Gravel et al. 2012 \citep{gravel_population_2012}. For Neandertal admixture where $m$, the admixture fraction, is typically low, the exponential assumption is satisfied \citep{liang_lengths_2014}.

\subsubsection{The Extended Pulse Model}\label{The Extended Pulse Model}

The extended pulse model as a generalization of the simple pulse model describes multi-generation admixture, where the random variable $T$ follows a continuous mixture distribution $\mathcal{D}_{T}$ instead of just being one value. Specifically that it follows a Gamma distribution with parameters $\Gamma(k,\frac{t_m}{k})$, hence

\begin{equation}
\label{eq:RV_extended_pulse}
  P(T)=\left\{
  \begin{array}{@{}ll@{}}
    \Gamma(k,\frac{t_m}{k}), & \text{if}\ T=t_{admix}\ \text{and}\ k\ \geq 1\\
    0, & \text{otherwise}
  \end{array}\right.
\end{equation} 

with,

\begin{equation}
\begin{split}
\label{eq:RV_extended_pulse_properties}
\mathbb{E}[t]&=t_{m} \\
Var[t]&=\frac{t_{m}^2}{k}  \\
\Gamma(k,\frac{t_m}{k})&= \frac{1}{\Gamma(k)(\frac{t_m}{k})^k}t^{k-1}e^{-t\frac{k}{t_m}}
\end{split}
\end{equation}

Where $t_{m}$ is the mean time of admixture and $Var[t]$ is defined as $(\frac{t_d}{4})^2$ with $t_{d}$ giving the admixure duration.
To get the likelihood function for the segment length distribution $P(L)$ under a multi-generation continuous extended pulse we have to solve the following integral:

\begin{equation}
\label{eq:Likelihood_function_simple_pulse_1}
    P(L=l) = \int_{0}^{\infty} \frac{1}{\Gamma(k)(\frac{t_m}{k})^k}t^{k-1}e^{-t\frac{k}{t_m}}\ t\ e^{-tl} \ dt 
\end{equation}

we can factor out all term not depending on $t$:

\begin{equation}
\label{eq:Likelihood_function_simple_pulse_2}
    P(L=l) = \frac{1}{\Gamma(k)(\frac{t_m}{k})^k}\ \int_{0}^{\infty}\ t^{k-1}e^{-t\frac{k}{t_m}}\ t\ e^{-tl} \ dt 
\end{equation}

we can re-write the integral into the  form of the known integral $\int_{0}^{\infty}\ x^n e^{-\alpha x} \ dx= \frac{\Gamma{n+1}}{\alpha^{n+1}}$

\begin{equation}
\begin{split}
\label{eq:Likelihood_function_simple_pulse_3}
    P(L=l) &= \frac{1}{\Gamma(k)(\frac{t_m}{k})^k}\ \int_{0}^{\infty}\ t^{k}e^{-(l+\frac{k}{t_m})t} \ dt \\ 
    P(L=l) &= \frac{1}{\Gamma(k)(\frac{t_m}{k})^k}\ \frac{\Gamma(k+1)}{l+\frac{k}{t_m})^{k+1}} 
\end{split}
\end{equation}

since $\frac{\Gamma(k+1)}{\Gamma(k)} =k$ we can re-write the likelihood to:

\begin{equation}
\begin{split}
\label{eq:Likelihood_function_simple_pulse_4}
    P(L=l) &= \frac{k}{(\frac{t_m}{k})^k \ (l+\frac{k}{t_m})^{k+1}} \\
    &= \frac{k(\frac{k}{t_m})^k} {(l+\frac{k}{t_{m})^{k+1}} } \\
    &= \frac{k^{k+1}} { t_{m}^k \ (l+\frac{k}{t_{m}})^{k+1}}  \\
    P(L=l) &= t_{m} \ \Bigg( \frac{k}{t_{m}(l+\frac{k}{t_{m}})}\Bigg)^{k+1}
\end{split}
\end{equation}

The result of equation \ref{eq:Likelihood_function_simple_pulse_4} is a density function known as $Lomax$ or $Pareto-II$-distribution. Hence, we can define the likelihood function for a Gamma distributed extended admixture pulse as being Lomax distributed with parameters $L(k+1,\frac{k}{t_{m}})$. 

with,

\begin{equation}
\label{eq:Expected_l_extended_pulse}
\mathbb{E}[l] = \frac{1}{t_{m}}
\end{equation}

So the expected segment length distribution for the mean time of admixture between the simple and extended pulse is equal (Eq. \ref{eq:Expected_l_simple_pulse} and  \ref{eq:Expected_l_extended_pulse}).



\hypertarget{refs}{}

\bibliography{References/MyLibraryATE}

\end{document}