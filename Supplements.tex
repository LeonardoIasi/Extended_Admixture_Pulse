\documentclass[]{article}
\usepackage{lmodern}
\usepackage{amssymb,amsmath}
\usepackage{ifxetex,ifluatex}
%\usepackage{fixltx2e} % provides \textsubscript
\usepackage{xr} % referencing external ducument
\ifnum 0\ifxetex 1\fi\ifluatex 1\fi=0 % if pdftex
  \usepackage[T1]{fontenc}
  \usepackage[utf8]{inputenc}
\else % if luatex or xelatex
  \ifxetex
    \usepackage{mathspec}
  \else
    \usepackage{fontspec}
  \fi
  \defaultfontfeatures{Ligatures=TeX,Scale=MatchLowercase}
\fi
% use upquote if available, for straight quotes in verbatim environments
\IfFileExists{upquote.sty}{\usepackage{upquote}}{}
% use microtype if available
\IfFileExists{microtype.sty}{%
\usepackage{microtype}
\UseMicrotypeSet[protrusion]{basicmath} % disable protrusion for tt fontshttps://de.overleaf.com/project/5e85b0680d0bed00011ea790
}{}
\usepackage[margin=1in]{geometry}
\usepackage{hyperref}
\hypersetup{unicode=true,
            pdftitle={Title: Human Neandertal Admixture dating limits MBE Supplements},
            pdfauthor={Leonardo Nicola Martin Iasi (Max Planck Institute for Evolutionary Anthropology, MPI EVA), Dr.~Benjamin Marco Peter (MPI EVA, benjamin\_peter@eva.mpg.de)},
            pdfborder={0 0 0},
            breaklinks=true}
\urlstyle{same}  % don't use monospace font for urls
 
%\usepackage{natbib}
%\bibliographystyle{References/my_abbrvnat}
%\setcitestyle{authoryear,open={(},close={)}}

\usepackage{graphicx,grffile}
\makeatletter
\def\maxwidth{\ifdim\Gin@nat@width>\linewidth\linewidth\else\Gin@nat@width\fi}
\def\maxheight{\ifdim\Gin@nat@height>\textheight\textheight\else\Gin@nat@height\fi}
\makeatother
% Scale images if necessary, so that they will not overflow the page
% margins by default, and it is still possible to overwrite the defaults
% using explicit options in \includegraphics[width, height, ...]{}

\setkeys{Gin}{width=\maxwidth,height=\maxheight,keepaspectratio}
\IfFileExists{parskip.sty}{%
\usepackage{parskip}
}{% else
\setlength{\parindent}{0pt}
\setlength{\parskip}{6pt plus 2pt minus 1pt}
}
\setlength{\emergencystretch}{3em}  % prevent overfull lines
\providecommand{\tightlist}{%
  \setlength{\itemsep}{0pt}\setlength{\parskip}{0pt}}
\setcounter{secnumdepth}{0}
% Redefines (sub)paragraphs to behave more like sections
\ifx\paragraph\undefined\else
\let\oldparagraph\paragraph
\renewcommand{\paragraph}[1]{\oldparagraph{#1}\mbox{}}
\fi
\ifx\subparagraph\undefined\else
\let\oldsubparagraph\subparagraph
\renewcommand{\subparagraph}[1]{\oldsubparagraph{#1}\mbox{}}
\fi



\usepackage{setspace}
\doublespacing
\usepackage[none]{hyphenat}
\usepackage{amsfonts}
\usepackage{amssymb}
\usepackage{graphicx}
\usepackage{float}
\usepackage{xcolor}

\floatplacement{figure}{H}
\begin{document}

\begin{titlepage}


    \vspace*{1cm}
        
        
    \begin{center}       
        \large
        \vspace{1cm}
        An extended admixture pulse model reveals the limits to the dating of Human-Neandertal introgression
        
       \vspace{1.0cm}
        \large
        Iasi, Leonardo N. M. \textsuperscript{1,2} and Peter , Benjamin M. \textsuperscript{1,3} \\ 
        
        \vspace{1.0cm}
            \Huge
            \textbf{Supplement Material}
    \end{center} 

            

\end{titlepage}
\section{Extended Pulse Model}

In this section, we describe in detail the derivation of the extended pulse model, where the gene flow over time is modeled as a Gamma distribution with shape parameter $k$ and scale parameter $\frac{k}{t_m}$. Here, $L_i$ is the length of a segment entered at time $T_i$. We assume each segment length at time $T_i$ to be described by an exponential distribution.

To get the likelihood function for the segment length distribution $P(L_i)$ under the extended pulse we have to solve the following integral:

\begin{equation}
\label{eq:Likelihood_function_extended_pulse_1}
    P(L_i=l) = \int_{0}^{\infty} \frac{1}{\Gamma(k)(\frac{t_m}{k})^k}t^{k-1}e^{-t\frac{k}{t_m}}\ t\ e^{-tl} \ dt 
\end{equation}

we can factor out all terms not depending on $t$:

\begin{equation}
\label{eq:Likelihood_function_extended_pulse_2}
    P(L=l) = \frac{1}{\Gamma(k)(\frac{t_m}{k})^k}\ \int_{0}^{\infty}\ t^{k-1}e^{-t\frac{k}{t_m}}\ t\ e^{-tl} \ dt 
\end{equation}

 
we can re-write the integral into the  form of the known integral $\int_{0}^{\infty}\ x^n e^{-\alpha x} \ dx= \frac{\Gamma{n+1}}{\alpha^{n+1}}$

\begin{equation}
\begin{split}
\label{eq:Likelihood_function_extended_pulse_3}
    P(L=l) &= \frac{1}{\Gamma(k)(\frac{t_m}{k})^k}\ \int_{0}^{\infty}\ t^{k}e^{-(l+\frac{k}{t_m})t} \ dt \\ 
    P(L=l) &= \frac{1}{\Gamma(k)(\frac{t_m}{k})^k}\ \frac{\Gamma(k+1)}{(l+\frac{k}{t_m})^{k+1}} 
\end{split}
\end{equation}

since $\frac{\Gamma(k+1)}{\Gamma(k)} =k$ we can re-write the likelihood to:



\begin{equation}
\begin{split}
\label{eq:Likelihood_function_extended_pulse_final}
    P(L=l) &= \frac{k}{(\frac{t_m}{k})^k \ (l+\frac{k}{t_m})^{k+1}} \\
    &= \frac{k(\frac{k}{t_m})^k} {(l+\frac{k}{t_{m}})^{k+1}}  \\
    &= \frac{k^{k+1}} { t_{m}^k \ (l+\frac{k}{t_{m}})^{k+1}}  \\
    &= t_{m} \ \Bigg( \frac{k}{t_{m}(l+\frac{k}{t_{m}})}\Bigg)^{k+1} \\
    P(L=l) &= t_{m}^{-k} \ \Bigg( \frac{k}{(l+\frac{k}{t_{m}})}\Bigg)^{k+1}
\end{split}
\end{equation}


\section{Supplement Figures}

\begin{figure}
\centering
\includegraphics{Admixture_Time_Inference_Paper_Draft_files/figure-latex/figS1-1.pdf}
\caption{\label{fig:figS1} Demographic models of Neandertal admixture with non-Africans used for the simulations. A) Simple demographic model used for ALD simulations with constant population sizes. B) Complex demographic model with substructure in Africa, where after an initial earlier split and isolation the structured population exchange migrants till the final split and additional gene flow between Africans and non-Africans after the Neandertal admixture. The  population sizes after the (final) split are taken fome MSMC/PSMC estimates for the respective populations.}
\end{figure}

\begin{figure}
\centering
\includegraphics{Admixture_Time_Inference_Paper_Draft_files/figure-latex/figS2-1.pdf}
\caption{\label{fig:figS2} Comparison of the standardized difference between true and estimated admixture time for simulations of all combinations of parameters: ascertainment scheme = LES/HES,  $d_{0}$ = 0.02/0.05 cM, demography = simple/complex, recombination = constant/variable and the gene flow model = simple/extended. Simulations under the simple pulse are indicated in purple, extended pulse in turquoise. Each simulation was repeated 100 times. Dotted horizontal line indicates no difference between true and estimated time.}
\end{figure}


\section{Supplement Tables}

\begin{table}[H]

\caption{\label{tab:tableS1} Mean, standart deviation, 5.5/94.5 compatibility interval of the posterior distribution for every parameter effect on the standardized difference between true and estimated admixture time.}
\centering
\begin{tabular}{l|r|r|r|r|r|r}
\hline
  & mean & sd & 5.5\% & 94.5\% & n\_eff & Rhat\\
\hline
a & 0.12 & 0.02 & 0.08 & 0.16 & 1220.94 & 1\\
\cline{1-7}
bG & -0.31 & 0.02 & -0.34 & -0.28 & 1559.84 & 1\\
\cline{1-7}
bR & -1.73 & 0.02 & -1.76 & -1.70 & 2112.69 & 1\\
\cline{1-7}
bD & -0.22 & 0.02 & -0.25 & -0.19 & 2036.66 & 1\\
\cline{1-7}
bm & 0.13 & 0.02 & 0.10 & 0.17 & 1657.66 & 1\\
\cline{1-7}
bA & -0.32 & 0.02 & -0.35 & -0.29 & 1721.62 & 1\\
\cline{1-7}
sigma & 0.46 & 0.01 & 0.44 & 0.47 & 2239.47 & 1\\
\hline
\end{tabular}
\end{table}


\begin{table}[H]

\caption{\label{tab:tableS2} Mean and 2.5/97.5 compatibility interval for every parameter estimated in the model fit}
\centering
\begin{tabular}[t]{l|l|l|l|l}
\hline
Model & Parameter & Estimate & 2.5 \% & 97.5 \%\\
\hline
Exponential & A & 0.018 & 0.018 & 0.018\\
\hline
Exponential & mean time & 1541 & 1487 & 1598\\
\hline
Exponential & c & 1e-05 & 6e-06 & 1.5e-05\\
\hline
Lomax (td=1) & A & 0.018 & 0.018 & 0.018\\
\hline
Lomax (td=1) & mean time & 1541 & 1487 & 1598\\
\hline
Lomax (td=1) & c & 1e-05 & 6e-06 & 1.5e-05\\
\hline
Lomax (td=100) & A & 0.018 & 0.018 & 0.018\\
\hline
Lomax (td=100) & mean time & 1541 & 1488 & 1599\\
\hline
Lomax (td=100) & c & 1e-05 & 6e-06 & 1.5e-05\\
\hline
Lomax (td=200) & A & 0.018 & 0.018 & 0.018\\
\hline
Lomax (td=200) & mean time & 1543 & 1490 & 1601\\
\hline
Lomax (td=200) & c & 1e-05 & 6e-06 & 1.5e-05\\
\hline
Lomax (td=400) & A & 0.018 & 0.018 & 0.018\\
\hline
Lomax (td=400) & mean time & 1552 & 1498 & 1610\\
\hline
Lomax (td=400) & c & 1e-05 & 6e-06 & 1.5e-05\\
\hline
Lomax (td=800) & A & 0.018 & 0.018 & 0.018\\
\hline
Lomax (td=800) & mean time & 1586 & 1531 & 1646\\
\hline
Lomax (td=800) & c & 1e-05 & 5e-06 & 1.4e-05\\
\hline
Lomax (td=1000) & A & 0.018 & 0.018 & 0.018\\
\hline
Lomax (td=1000) & mean time & 1613 & 1557 & 1673\\
\hline
Lomax (td=1000) & c & 9e-06 & 5e-06 & 1.3e-05\\
\hline
Lomax (td=1500) & A & 0.018 & 0.018 & 0.018\\
\hline
Lomax (td=1500) & mean time & 1713 & 1654 & 1777\\
\hline
Lomax (td=1500) & c & 7e-06 & 3e-06 & 1.1e-05\\
\hline
Lomax (td=2000) & A & 0.018 & 0.018 & 0.018\\
\hline
Lomax (td=2000) & mean time & 1875 & 1809 & 1945\\
\hline
Lomax (td=2000) & c & 4e-06 & 0 & 8e-06\\
\hline
Lomax (td=2500) & A & 0.018 & 0.018 & 0.018\\
\hline
Lomax (td=2500) & mean time & 2129 & 2055 & 2208\\
\hline
Lomax (td=2500) & c & -1e-06 & -5e-06 & 4e-06\\
\hline
\end{tabular}
\end{table}

\end{document}